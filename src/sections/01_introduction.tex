In this manuscript,
we obtain formulas for sums of powers via Newton’s interpolation formula
based on backward finite differences of powers.
The idea to derive sums of powers using difference operator and Newton's series is quite generic,
thus, formulas for sums of powers using forward and central differences can be found
in the works~\cite{kolosov_2025_18102534, kolosov_2025_18098442}.

Define multifold sums of powers in Knuth's~\cite{knuth1993johann} notation
\begin{align*}
    \KnuthRFoldSum{0}{n}{m}   &= n^m \\
    \KnuthRFoldSum{1}{n}{m}   &= \KnuthRFoldSum{0}{1}{m} + \KnuthRFoldSum{0}{2}{m} + \cdots + \KnuthRFoldSum{0}{n}{m} \\
    \KnuthRFoldSum{r+1}{n}{m} &= \KnuthRFoldSum{r}{1}{m} + \KnuthRFoldSum{r}{2}{m} + \cdots + \KnuthRFoldSum{r}{n}{m}
\end{align*}
The book Interpolation by Steffensen~\cite[chapter 2, eq. (19)]{steffensen1927interpolation}
gives Newton's formula for backward differences evaluated
in zero $f(x) = \sum_{k=0}^{n} \binom{x+k-1}{k} \nabla^{k} f(0)$.

In general,
\begin{proposition}[Newton formula via backward differences]
    \label{prop:newtons-formula-via-backward-difference}
    \begin{mdframed}
        \begin{align*}
            f(x) &= \sum_{k=0}^{n} \binom{x-a+k-1}{k} \nabla^{k} f(a)
        \end{align*}
        where $\nabla^{k} f(a) = \sum_{j=0}^{k} (-1)^{j} \binom{k}{j} f(a-j)$.
    \end{mdframed}
\end{proposition}
Thus, by setting $f(n)=n^m$
\begin{align*}
    n^m &= \sum_{j=0}^{m} \binom{n-t+j-1}{j} \nabla^{j} t^{m},
\end{align*}
where $\nabla^{j} t^{m} = \sum_{k=0}^{j} (-1)^{k} \binom{j}{k} (t-k)^{m}$.
Therefore, ordinary sums of powers is equivalent to
\begin{align*}
    \KnuthRFoldSum{1}{n}{m} &= \sum_{j=0}^{m} \nabla^{j} t^{m} \sum_{k=1}^{n} \binom{k-t+j-1}{j}
\end{align*}
We notice that the sum $\sum_{k=1}^{n} \binom{k-t+j-1}{j}$ is a good candidate for hockey stick identity for binomial coefficients
$\sum_{k=0}^{n} \binom{k}{j} = \binom{n+1}{j+1}$.
Thus, by setting $a=j-t$ and $b=j-t-1+n$, we get
\begin{align*}
    \sum_{k=1}^{n} \binom{-t+j-1+k}{j} = \sum_{m=j-t}^{j-t-1+n} \binom{m}{j}
\end{align*}
Thus,
\begin{align*}
    \sum_{k=1}^{n} \binom{-t+j-1+k}{j} = \binom{j-t+n}{j+1} - \binom{j-t}{j+1}
\end{align*}
Because,
\begin{lemma} [Generalized hockey stick identity]
    \label{lem:generalized-hockey-stick-identity}
    \begin{align*}
        \sum_{m=a}^{b} \binom{m}{j} = \binom{b+1}{j+1} - \binom{a}{j+1}
    \end{align*}
\end{lemma}
Applying the identity for binomial coefficients $\binom{-k}{j} = (-1)^j \binom{j+k-1}{j}$, we obtain
\begin{proposition} [Ordinary sums of powers via backward differences]
    \label{prop:ordinary-sums-of-powers-via-backward-differences}
    \begin{mdframed}
        For non-negative integers $n,m$ and an arbitrary integer $t$
        \begin{align*}
            \KnuthRFoldSum{1}{n}{m} &= \sum_{j=0}^{m} \nabla^{j} t^{m} \left[ (-1)^j \binom{t}{j+1} + \binom{j-t+n}{j+1} \right]
        \end{align*}
    \end{mdframed}
\end{proposition}
For example, by setting $t=2$ and $m=1,2,3,4$, we get formulas for sums of cubes
\begin{align*}
    \KnuthRFoldSum{1}{n}{1} &=
    2\left[ -\binom{2}{1} + \binom{n-2}{1} \right]
    + 1\left[ \binom{2}{2} + \binom{n-1}{2} \right],
\end{align*}
\begin{align*}
    \KnuthRFoldSum{1}{n}{2}
    &=
    4\left[ -\binom{2}{1} + \binom{n-2}{1} \right]
    + 3\left[ \binom{2}{2} + \binom{n-1}{2} \right] \\
    &\quad
    + 2\left[ -\binom{2}{3} + \binom{n}{3} \right].
\end{align*}
\begin{align*}
    \KnuthRFoldSum{1}{n}{3}
    &=
    8\left[ -\binom{2}{1} + \binom{n-2}{1} \right]
    + 7\left[ \binom{2}{2} + \binom{n-1}{2} \right] \\
    &\quad
    + 6\left[ -\binom{2}{3} + \binom{n}{3} \right]
    + 6\left[ \binom{2}{4} + \binom{n+1}{4} \right].
\end{align*}
\begin{align*}
    \KnuthRFoldSum{1}{n}{4}
    &=
    16\left[ -\binom{2}{1} + \binom{n-2}{1} \right]
    + 15\left[ \binom{2}{2} + \binom{n-1}{2} \right] \\
    &\quad
    + 14\left[ -\binom{2}{3} + \binom{n}{3} \right]
    + 12\left[ \binom{2}{4} + \binom{n+1}{4} \right] \\
    &\quad
    + 24\left[ -\binom{2}{5} + \binom{n+2}{5} \right].
\end{align*}
\begin{itemize}
    \item For $t=0$ the coefficients are $1, 0, 1, 0, -1, 2, 0, 1, -6, 6,\ldots$ and registered in the OEIS~\cite{sloane2003line} as \href{https://oeis.org/A278075}{\texttt{A278075}}.
    \item For $t=1$ the coefficients are $1, 1, 1, 1, 1, 2, 1, 1, 0, 6,\ldots$ and registered in the OEIS~\cite{sloane2003line} as \href{https://oeis.org/A389570}{\texttt{A389570}}.
    \item For $t=2$ the coefficients are $1, 2, 1, 4, 3, 2, 8, 7, 6, 6,\ldots$ and registered in the OEIS~\cite{sloane2003line} as \href{https://oeis.org/A391068}{\texttt{A391068}}.
    \item For $t=3$ the coefficients are $1, 3, 1, 9, 5, 2, 27, 19, 12, 6,\ldots$ and registered in the OEIS~\cite{sloane2003line} as \href{https://oeis.org/A391210}{\texttt{A391210}}.
\end{itemize}
\begin{lemma} [Backward differences in Eulerian numbers]
    \label{lem:backward-differences-in-eulerian-numbers}
    \begin{align*}
        \Delta^{j} t^{m} = \sum_{k=0}^{m} \binom{t+k-j}{m-j} \eulerian{m}{k}
    \end{align*}
    \begin{proof}
        By Worpitzky identity $t^{m} = \sum_{k=0}^{m} \binom{t+k}{m} \eulerian{m}{k}$ and binomial recurrence $\binom{n+1}{k} = \binom{n}{k} + \binom{n}{k-1}$, see~\cite{Worpitzky1883}.
    \end{proof}
\end{lemma}
Thus, let be a formula for ordinary sums of powers in terms of Eulerian numbers $\eulerian{m}{k}$
\begin{proposition} [Ordinary sums of powers in Eulerian numbers]
    \label{prop:ordinary-sums-of-powers-in-eulerian-numbers}
    \begin{mdframed}
        For non-negative integers $n,m$ and an arbitrary integer $t$
        \begin{align*}
            \KnuthRFoldSum{1}{n}{m} &= \sum_{j=0}^{m} \sum_{k=0}^{m} \left[ (-1)^j \binom{t}{j+1} + \binom{j-t+n}{j+1} \right] \binom{t+k-j}{m-j} \eulerian{m}{k}
        \end{align*}
    \end{mdframed}
\end{proposition}
Remarkable that having $t=0$ formula for sums of powers turns into double binomial view
\begin{proposition} [Ordinary Eulerian sums of powers in zero]
    \begin{mdframed}
        For non-negative integers $n,m$
        \begin{align*}
            \KnuthRFoldSum{1}{n}{m} &= \sum_{j=0}^{m} \sum_{k=0}^{m} \binom{j+n}{j+1}  \binom{k-j}{m-j} \eulerian{m}{k}
        \end{align*}
    \end{mdframed}
\end{proposition}
\begin{lemma} [Backward differences in Stirling numbers]
    \label{lem:backward-differences-in-stirling-numbers}
    \begin{align*}
        \nabla^{j} t^{m} = \sum_{k=j}^{m} \binom{t-j}{k-j} \stirlingii{m}{k} k!
    \end{align*}
    \begin{proof}
        By the identity $t^m  = \sum_{k=0}^{m} \binom{t}{k} \stirlingii{m}{k} k!$ and binomial recurrence $\binom{n+1}{k} = \binom{n}{k} + \binom{n}{k-1}$.
    \end{proof}
\end{lemma}
Thus, let be a formula for ordinary sums of powers in terms of Stirling numbers $\stirlingii{m}{k}$
\begin{proposition} [Ordinary sums of powers in Stirling numbers]
    \label{prop:ordinary-sums-of-powers-in-stirling-numbers}
    \begin{mdframed}
        For non-negative integers $n,m$ and an arbitrary integer $t$
        \begin{align*}
            \KnuthRFoldSum{1}{n}{m} &= \sum_{j=0}^{m} \sum_{k=j}^{m} \left[ (-1)^j \binom{t}{j+1} + \binom{j-t+n}{j+1} \right] \binom{t-j}{k-j} \stirlingii{m}{k} k!
        \end{align*}
    \end{mdframed}
\end{proposition}
By setting $t=0$ yields
\begin{proposition} [Ordinary Stirling sums of powers in zero]
    \begin{mdframed}
        For non-negative integers $n,m$
        \begin{align*}
            \KnuthRFoldSum{1}{n}{m} &= \sum_{j=0}^{m} \sum_{k=j}^{m} \binom{j+n}{j+1} \binom{-j}{k-j} \stirlingii{m}{k} k!
        \end{align*}
    \end{mdframed}
\end{proposition}
Formula for double sums of powers can be derived in a similar manner,
by applying summation to the proposition~\eqref{prop:ordinary-sums-of-powers-via-backward-differences},
which in turn implies generalized hockey-stick identity, thus
\begin{align*}
    \KnuthRFoldSum{2}{n}{m} &= \sum_{j=0}^{m} \nabla^{j} t^{m} \left[ (-1)^j \binom{t}{j+1} \sum_{k=1}^{n} 1 + \sum_{k=1}^{n} \binom{j-t+k}{j+1} \right]
\end{align*}
By applying generalized hockey stick identity~\eqref{lem:generalized-hockey-stick-identity}, we obtain
\begin{align*}
    \sum_{k=1}^{n} \binom{j-t+k}{j+1} = \sum_{k=j-t+1}^{j-t+n} \binom{k}{j+1} = \binom{j-t+n+1}{j+2} - \binom{j-t+1}{j+2}
\end{align*}
Therefore,
\begin{align*}
    \KnuthRFoldSum{2}{n}{m}
    &= \sum_{j=0}^{m} \nabla^{j} t^{m} \left[ (-1)^j \binom{t}{j+1} n + \left( \binom{j-t+n+1}{j+2} - \binom{j-t+1}{j+2} \right) \right]
\end{align*}
By applying the identity for negative binomial coefficients $\binom{-k}{j} = (-1)^j \binom{j+k-1}{j}$, we get
\begin{align*}
    \binom{-(t-j-1)}{j+2} = (-1)^{j+2} \binom{t}{j+2}
\end{align*}
Hence,
\begin{proposition} [Double sums of powers via backward differences]
    \label{prop:double-sums-of-powers-via-backward-differences}
    \begin{mdframed}
        For non-negative integers $n,m$ and an arbitrary integer $t$
        \begin{align*}
            \KnuthRFoldSum{2}{n}{m}
            &= \sum_{j=0}^{m} \nabla^{j} t^{m} \left[ (-1)^j \binom{t}{j+1} n + (-1)^{j+1} \binom{t}{j+2} n^0 +  \binom{j-t+n+1}{j+2} \right]
        \end{align*}
    \end{mdframed}
\end{proposition}
In general,
\begin{theorem} [Multifold sums of powers via backward difference]
    \label{theorem:multifold-sums-of-powers-via-backward-difference}
    \begin{mdframed}
        For non-negative integers $r,n,m$ and an arbitrary integer $t$
        \begin{align*}
            \KnuthRFoldSum{r}{n}{m} = \sum_{j=0}^{m} \nabla^{j} t^{m} \left[ \binom{j-t+n+r-1}{j+r} + \sum_{s=0}^{r-1} (-1)^{j+s} \binom{t}{j+s+1} \KnuthRFoldSum{r-1-s}{n}{0} \right]
        \end{align*}
    \end{mdframed}
\end{theorem}
We may observe that
\begin{proposition}[Multifold sum of zero powers]
    \label{prop:multifold-sum-of-zero-powers}
    For integers $r$ and $n$
    \begin{align*}
        \KnuthRFoldSum{r}{n}{0} = \binom{r+n-1}{r}
    \end{align*}
    \begin{proof}
        By hockey stick identity $\sum_{k=0}^{t} \binom{j+k}{j} = \binom{j+t+1}{j+1}$.
    \end{proof}
\end{proposition}
By $\KnuthRFoldSum{r-1-s}{n}{0} = \binom{r-s+n-2}{r-s-1}$, we get
\begin{proposition} [Multifold sums of powers binomial form]
    \label{prop:multifold-sums-of-powers-via-backward-difference-binomial-form}
    \begin{mdframed}
        For non-negative integers $r,n,m$ and an arbitrary integer $t$
        \begin{align*}
            \KnuthRFoldSum{r}{n}{m} = \sum_{j=0}^{m} \nabla^{j} t^{m} \left[ \binom{j-t+n+r-1}{j+r} + \sum_{s=0}^{r-1} (-1)^{j+s} \binom{t}{j+s+1} \binom{r-s+n-2}{r-s-1} \right]
        \end{align*}
    \end{mdframed}
\end{proposition}
By setting $r \rightarrow r+1$
\begin{corollary} [Multifold sums of powers binomial form shifted]
    \begin{mdframed}
        For non-negative integers $r,n,m$ and an arbitrary integer $t$
        \begin{align*}
            \KnuthRFoldSum{r+1}{n}{m} = \sum_{j=0}^{m} \nabla^{j} t^{m} \left[ \binom{j-t+n+r}{j+r+1} + \sum_{s=0}^{r} (-1)^{j+s} \binom{t}{j+s+1} \binom{r-s+n-1}{r-s} \right]
        \end{align*}
    \end{mdframed}
\end{corollary}
By lemma~\eqref{lem:backward-differences-in-stirling-numbers}, we get formula for multifold sums of powers
in terms of Stirling numbers of the second kind
\begin{proposition}[Multifold sums of powers in Stirling numbers]
    \label{prop:multifold-sums-of-powers-in-stirling-numbers}
    \begin{mdframed}
        For non-negative integers $r,n,m$ and an arbitrary integer $t$
        \begin{align*}
            \KnuthRFoldSum{r+1}{n}{m}
            = \sum_{j=0}^{m} \sum_{k=j}^{m} \left[ \binom{j-t+n+r}{j+r+1} + \sum_{s=0}^{r} (-1)^{j+s} \binom{t}{j+s+1} \binom{r-s+n-1}{r-s} \right] \binom{t-j}{k-j} \stirlingii{m}{k} k!
        \end{align*}
    \end{mdframed}
\end{proposition}
By lemma~\eqref{lem:backward-differences-in-eulerian-numbers}, we get formula for multifold sums of powers
in terms of Eulerian numbers
\begin{proposition}[Multifold sums of powers in Eulerian numbers]
    \label{prop:multifold-sums-of-powers-in-eulerian-numbers}
    \begin{mdframed}
        For non-negative integers $r,n,m$ and an arbitrary integer $t$
        \begin{align*}
            \KnuthRFoldSum{r+1}{n}{m}
            = \sum_{j=0}^{m} \sum_{k=0}^{m} \left[ \binom{j-t+n+r}{j+r+1} + \sum_{s=0}^{r} (-1)^{j+s} \binom{t}{j+s+1} \binom{r-s+n-1}{r-s} \right] \binom{t+k-j}{m-j} \eulerian{m}{k}
        \end{align*}
    \end{mdframed}
\end{proposition}
