\documentclass[12pt,letterpaper,oneside,reqno]{amsart}
\usepackage{amsmath}
\usepackage{amssymb}
\usepackage{amsthm}
\usepackage{float}
\usepackage[font=small,labelfont=bf]{caption}
\usepackage[unicode,pdfpagelabels,hyperindex,colorlinks=true,linkcolor=red,urlcolor=blue,citecolor=red]{hyperref}
\usepackage{graphicx}
\emergencystretch=1em
\usepackage{array}
\usepackage{enumitem}
\usepackage{etoolbox}
\usepackage{physics}
\usepackage{booktabs}
\usepackage{mdframed}

% margins and layout
\linespread{1.7}
\usepackage[left=1in,right=1in,bottom=1in,top=1in]{geometry}
\apptocmd{\sloppy}{\hbadness 10000\relax}{}{}
\raggedbottom

\newcommand \coeffA [3][A] {{\mathbf{#1}} \sb{#2,#3}}
\newcommand \polynomialP [4][P]{{\mathbf{#1}}\sp{#2} \sb{#3}(#4)}
\newcommand \bernoulli [2][B] {{#1}\sb{#2}}

%\newcommand \anglePower [2]{\langle #1 \rangle \sp{#2}}
%\newcommand \curvePower [2]{\{#1\}\sp{#2}}

% central factorials and related symbols
\newcommand \centralFactorial [2] {#1^{[#2]}}
\newcommand \fallingFactorial [2] {\left(#1 \right)^{\underline{#2}}}

\newcommand{\stirlingii}{\genfrac{\{}{\}}{0pt}{}}
\newcommand{\eulerian}[2]{\genfrac{\langle}{\rangle}{0pt}{}{#1}{#2}} % Eulerian

\newcommand{\KnuthRFoldSum}[3]{\Sigma^{#1}\,{#2}^{#3}}

% ~~~ Rascal numbers ~~~
%\newcommand \rascalNumber [3] {\binom{#1}{#2}_{#3}}
%\newcommand \north[0] {\mathbf{North}}
%\newcommand \south[0] {\mathbf{South}}
%\newcommand \west[0] {\mathbf{West}}
%\newcommand \east[0] {\mathbf{East}}

% ~~~~ 1-q pascal notation~~~~
%\newcommand{\genstirlingI}[3]{%
%    \genfrac{[}{]}{0pt}{#1}{#2}{#3}%
%}
%\newcommand{\genstirlingII}[3]{%
%    \genfrac{\{}{\}}{0pt}{#1}{#2}{#3}%
%}
%\newcommand{\oneQBinomial}[3]{\genstirlingI{}{#1}{#2}^{#3}}


\newtheorem{theorem}{Theorem}[section]
\newtheorem{corollary}[theorem]{Corollary}
\newtheorem{proposition}[theorem]{Proposition}
\newtheorem{observation}[theorem]{Observation}
\newtheorem{lemma}[theorem]{Lemma}
\newtheorem{claim}[theorem]{Claim}
\newtheorem{example}[theorem]{Example}
\newtheorem{conjecture}[theorem]{Conjecture}
\newtheorem{definition}[theorem]{Definition}
\newtheorem{question}[theorem]{Question}
\newtheorem{remark}[theorem]{Remark}
\newtheorem{assumption}[theorem]{Assumption}

% free foot note
\let\svthefootnote\thefootnote
\newcommand\freefootnote[1]{%
    \let\thefootnote\relax%
    \footnotetext{#1}%
    \let\thefootnote\svthefootnote%
}

%\numberwithin{equation}{section}

\title[Sums of powers via backward finite differences and Newton's formula]
{Sums of powers via backward finite differences and Newton's formula}
\author[Petro Kolosov]{Petro Kolosov}
\date{\today}

% metadata
\input{metadata/msc2010.tex}
\input{metadata/keywords.tex}
\input{metadata/hypersetup.tex}
\address{DevOps Engineer}
\email{kolosovp94@gmail.com}
\urladdr{https://kolosovpetro.github.io}

\begin{document}

    \begin{abstract}
        We obtain formulas for sums of powers via Newton's interpolation
formula based on backward finite differences of powers.
In addition, we note that backward differences are closely
related to Eulerian numbers, and Stirling numbers of the second kind.
Thus, we express formulas for sums of powers in terms of Eulerian numbers,
and Stirling numbers of the second kind.

    \end{abstract}

    \maketitle

    \tableofcontents

    \freefootnote{DOI: \url{https://doi.org/10.5281/zenodo.18118011}}


    \section{Introduction and main results}
    \label{sec:introduction-and-main-results}
    In this manuscript, we derive formulas for multifold sums of powers
by utilizing Newton's interpolation formula.
Furthermore, we provide formulas for multifold sums of powers in terms
of Stirling numbers of the second kind $\stirlingii{n}{k}$
and Eulerian numbers $\eulerian{n}{k}$.
The main idea is to utilize Newton's interpolation formula
combined with hockey stick identities to
derive formulas for multifold sums of powers.
In this work, we use Newton's interpolation formula
in backward finite differences.
However, Newton's formula approach for sums of powers is quite generic,
as such it does not necessary requires to bind to forward difference operator
specifically. Thus, formulas for sums of powers using forward and central
differences can be found
in~~\cite{kolosov_2025_18102534, kolosov_2025_18098442}, respectively.
Define multifold sums of powers in
Knuth's~\cite{knuth1993johann}
notation
\begin{proposition} [Multifold sums of powers recurrence]
  \label{prop:multifold-sums-of-powers-recurrence}
  For integers $r,n,m \geq 0$,
  \begin{align*}
    \KnuthRFoldSum{0}{n}{m}   &= n^m, \\
    \KnuthRFoldSum{1}{n}{m}   &= \KnuthRFoldSum{0}{1}{m}
    + \KnuthRFoldSum{0}{2}{m} + \cdots + \KnuthRFoldSum{0}{n}{m}, \\
    \KnuthRFoldSum{r+1}{n}{m} &= \KnuthRFoldSum{r}{1}{m}
    + \KnuthRFoldSum{r}{2}{m} + \cdots + \KnuthRFoldSum{r}{n}{m}.
  \end{align*}
\end{proposition}
Steffensen
gives Newton's formula for backward differences evaluated
at zero
\begin{align*}
  f(x) = \sum_{k=0}^{n} \binom{x+k-1}{k} \nabla^{k} f(0)
\end{align*}
in his book
Interpolation~\cite[chapter 2, eq. (19)]{steffensen1927interpolation}.
Thus, Newton's formula for backward differences evaluated
at arbitrary integer $a$ follows.
\begin{proposition}[Newton formula in backward differences]
  Let $f\colon \mathbb{Z} \to \mathbb{C}$ be a function,
  and let $a \in \mathbb{Z}$.
  Then, for all $x \in \mathbb{Z}$,
  \label{prop:newtons-formula-via-backward-difference}
  \begin{align*}
    f(x) &= \sum_{k=0}^{n} \binom{x-a+k-1}{k} \nabla^{k} f(a),
  \end{align*}
  where the $k$-th backward finite difference of $f$ evaluated
  at $a$ is defined by
  \begin{align*}
    \nabla^{k} f(a) = \sum_{j=0}^{k} (-1)^{j} \binom{k}{j} f(a-j).
  \end{align*}
\end{proposition}
Thus, by setting $f(n)=n^m$, we get interpolation of powers
via backward differences evaluated at arbitrary point $t$.
\begin{corollary}[Backward Newton's formula for powers]
  \label{corr:backward-netwons-formula-for-powers}
  For integers $n,m \geq 0$,
  \begin{align*}
    n^m &= \sum_{j=0}^{m} \binom{n-t+j-1}{j} \nabla^{j} t^{m},
  \end{align*}
  where $\nabla^{j} t^{m} = \sum_{k=0}^{j} (-1)^{k} \binom{j}{k} (t-k)^{m}$.
\end{corollary}
Therefore, by Corollary~\eqref{corr:backward-netwons-formula-for-powers},
identity for sums of powers follows.
\begin{align*}
  \KnuthRFoldSum{1}{n}{m}
  = \sum_{j=0}^{m} \nabla^{j} t^{m} \sum_{k=1}^{n} \binom{k-t+j-1}{j}.
\end{align*}
We notice that the sum $\sum_{k=1}^{n} \binom{k-t+j-1}{j}$
is a good candidate for hockey stick identity for binomial coefficients
$\sum_{k=0}^{n} \binom{k}{j} = \binom{n+1}{j+1}$.
Thus, by setting $a=j-t$ and $b=j-t-1+n$, we get,
\begin{align*}
  \sum_{k=1}^{n} \binom{k-t+j-1}{j}
  = \sum_{m=j-t}^{j-t-1+n} \binom{m}{j}
  = \binom{j-t+n}{j+1} - \binom{j-t}{j+1}.
\end{align*}
% which implies,
% \begin{align*}
%   \sum_{k=1}^{n} \binom{-t+j-1+k}{j} = \binom{j-t+n}{j+1} - \binom{j-t}{j+1}.
% \end{align*}
The closed form above is a partial case of Generalized hockey stick identity.
That is,
\begin{lemma} [Generalized hockey stick identity]
  \label{lem:generalized-hockey-stick-identity}
  For integers $a \leq b$ and $j$,
  \begin{align*}
    \sum_{m=a}^{b} \binom{m}{j} = \binom{b+1}{j+1} - \binom{a}{j+1}.
  \end{align*}
  \begin{proof}
    We have,
    \begin{align*}
      \tsum_{k=a}^{b} \tbinom{k}{j} = \tbinom{a}{j} + \tbinom{a+1}{j}
      + \cdots + \tbinom{b}{j},
    \end{align*}
    which implies,
    \begin{align*}
      \tsum_{k=a}^{b} \tbinom{k}{j} =
      \left( \tsum_{k=0}^{b} \tbinom{k}{j} \right)
      -
      \left(\tsum_{k=0}^{a-1} \tbinom{k}{j}\right).
    \end{align*}
    Thus, by hockey stick identity
    $\tsum_{k=0}^{n} \binom{k}{j} = \binom{n+1}{j+1}$,
    we get,
    \begin{align*}
      \tsum_{k=a}^{b} \tbinom{k}{j}
      =
      \left(
        \tsum_{k=0}^{b} \tbinom{k}{j}
      \right)
      -
      \left(
        \tsum_{k=0}^{a-1} \tbinom{k}{j}
      \right)
      = \tbinom{b+1}{j+1} - \tbinom{a}{j+1}.
    \end{align*}
    This completes the proof.
  \end{proof}
\end{lemma}
\begin{lemma}[Shifted hockey stick identity]
  \label{lem:shifted-hockey-stick-identity}
  For integers $j,r \geq 0$, and an arbitrary integer $t$,
  \begin{align*}
    \sum_{k=1}^{n} \binom{j-t+k+r-1}{j+r}
    =\sum_{a=j-t+r}^{j-t+n+r-1} \binom{a}{j+r}
    = \binom{j-t+n+r}{j+r+1} - \binom{j-t+r}{j+r+1}.
  \end{align*}
\end{lemma}
\begin{proposition} [Backward sums of powers]
  \label{prop:backward-sums-of-powers}
  For integers $n,m \geq 0$, and an arbitrary integer $t$,
  \begin{align*}
    \KnuthRFoldSum{1}{n}{m} &= \sum_{j=0}^{m} \nabla^{j} t^{m}
    \left[
      \binom{j-t+n}{j+1} - \binom{j-t}{j+1}
    \right].
  \end{align*}
\end{proposition}
Recall the negation property of binomial coefficients
$\binom{-k}{j} = (-1)^j \binom{j+k-1}{j}$.
Thus, formula for negated backward sums of powers follows.
\begin{proposition} [Negated backward sums of powers]
  \label{prop:negated-backward-sums-of-powers}
  For integers $n,m \geq 0$, and an arbitrary integer $t$,
  \begin{align*}
    \KnuthRFoldSum{1}{n}{m} &= \sum_{j=0}^{m} \nabla^{j} t^{m}
    \left[
      \binom{j-t+n}{j+1} + (-1)^j \binom{t}{j+1}
    \right].
  \end{align*}
\end{proposition}
For example, by setting $t=2$ and $m=1$,
we get,
\begin{align*}
  \KnuthRFoldSum{1}{n}{1} &=
  2\left[ -\tbinom{2}{1} + \tbinom{n-2}{1} \right]
  + 1\left[ \tbinom{2}{2} + \tbinom{n-1}{2} \right].
\end{align*}
For $m=2$:
\begin{align*}
  \KnuthRFoldSum{1}{n}{2}
  =
  4\left[ -\tbinom{2}{1} + \tbinom{n-2}{1} \right]
  + 3\left[ \tbinom{2}{2} + \tbinom{n-1}{2} \right]
  + 2\left[ -\tbinom{2}{3} + \tbinom{n}{3} \right].
\end{align*}
For $m=3$:
\begin{align*}
  \KnuthRFoldSum{1}{n}{3}
  =
  8\left[ -\tbinom{2}{1} + \tbinom{n-2}{1} \right]
  + 7\left[ \tbinom{2}{2} + \tbinom{n-1}{2} \right]
  + 6\left[ -\tbinom{2}{3} + \tbinom{n}{3} \right]
  + 6\left[ \tbinom{2}{4} + \tbinom{n+1}{4} \right].
\end{align*}
For $m=4$:
\begin{align*}
  \KnuthRFoldSum{1}{n}{4}
  &=
  16\left[ -\tbinom{2}{1} + \tbinom{n-2}{1} \right]
  + 15\left[ \tbinom{2}{2} + \tbinom{n-1}{2} \right]
  + 14\left[ -\tbinom{2}{3} + \tbinom{n}{3} \right]
  + 12\left[ \tbinom{2}{4} + \tbinom{n+1}{4} \right] \\
  & + 24\left[ -\tbinom{2}{5} + \tbinom{n+2}{5} \right].
\end{align*}
The coefficients $1, 2, 1, 4, 3, 2, 8, 7, 6, 6,\ldots$
is the sequence \href{https://oeis.org/A391068}{\texttt{A391068}}
in the OEIS~\cite{sloane2003line}.
% \begin{itemize}
%   \item For $t=0$ the coefficients are $1, 0, 1, 0, -1, 2, 0, 1, -6, 6,\ldots$ and registered in the OEIS~\cite{sloane2003line} as \href{https://oeis.org/A278075}{\texttt{A278075}}.
%   \item For $t=1$ the coefficients are $1, 1, 1, 1, 1, 2, 1, 1, 0, 6,\ldots$ and registered in the OEIS~\cite{sloane2003line} as \href{https://oeis.org/A389570}{\texttt{A389570}}.
%   \item For $t=2$ the coefficients are $1, 2, 1, 4, 3, 2, 8, 7, 6, 6,\ldots$ and registered in the OEIS~\cite{sloane2003line} as \href{https://oeis.org/A391068}{\texttt{A391068}}.
%   \item For $t=3$ the coefficients are $1, 3, 1, 9, 5, 2, 27, 19, 12, 6,\ldots$ and registered in the OEIS~\cite{sloane2003line} as \href{https://oeis.org/A391210}{\texttt{A391210}}.
% \end{itemize}
The following Lemma connects the Eulerian numbers $\eulerian{n}{k}$ and backward
differences of powers.
\begin{lemma} [Backward differences in Eulerian numbers]
  For integers $j,m \geq 0$, and an arbitrary integer $t$,
  \label{lem:backward-differences-in-eulerian-numbers}
  \begin{align*}
    \nabla^{j} t^{m} = \sum_{k=0}^{m} \binom{t+k-j}{m-j} \eulerian{m}{k}.
  \end{align*}
  \begin{proof}
    By Worpitzky identity
    $t^{m} = \sum_{k=0}^{m} \binom{t+k}{m} \eulerian{m}{k}$,
    and by binomial recurrence
    $\binom{n+1}{k} = \binom{n}{k} + \binom{n}{k-1}$,
    see~\cite{Worpitzky1883}.
  \end{proof}
\end{lemma}
Thus, we get formula for ordinary sums of powers in terms
of Eulerian numbers $\eulerian{m}{k}$.
\begin{proposition} [Ordinary sums of powers in Eulerian numbers]
  \label{prop:ordinary-sums-of-powers-in-eulerian-numbers}
  For non-negative integers $n,m$ and an arbitrary integer $t$
  \begin{align*}
    \KnuthRFoldSum{1}{n}{m} &= \sum_{j=0}^{m} \sum_{k=0}^{m}
    \left[
      (-1)^j \binom{t}{j+1} + \binom{j-t+n}{j+1}
    \right] \binom{t+k-j}{m-j} \eulerian{m}{k}
  \end{align*}
\end{proposition}
It is quite
remarkable that
having $t=0$ yields formula for sums of powers in double binomial form,
involving Eulerian numbers $\eulerian{m}{k}$.
\begin{proposition} [Ordinary Eulerian sums of powers in zero]
  \label{prop:ordinary-sums-of-eulerian-numbers-in-zero}
  For non-negative integers $n,m$,
  \begin{align*}
    \KnuthRFoldSum{1}{n}{m}
    = \sum_{j=0}^{m}
    \sum_{k=0}^{m} \binom{j+n}{j+1}  \binom{k-j}{m-j} \eulerian{m}{k}.
  \end{align*}
\end{proposition}
\begin{lemma} [Backward differences in Stirling numbers]
  For integers $j,m \geq 0$, and an arbitrary integer $t$,
  \label{lem:backward-differences-in-stirling-numbers}
  \begin{align*}
    \nabla^{j} t^{m} = \sum_{k=j}^{m} \binom{t-j}{k-j} \stirlingii{m}{k} k!
  \end{align*}
  \begin{proof}
    By the identity $t^m  = \sum_{k=0}^{m} \binom{t}{k} \stirlingii{m}{k} k!$
    and binomial recurrence $\binom{n+1}{k} = \binom{n}{k} + \binom{n}{k-1}$.
  \end{proof}
\end{lemma}
Thus, let be a formula for ordinary sums of powers in terms of Stirling
numbers $\stirlingii{m}{k}$
\begin{proposition} [Ordinary sums of powers in Stirling numbers]
  \label{prop:ordinary-sums-of-powers-in-stirling-numbers}
  For non-negative integers $n,m$ and an arbitrary integer $t$,
  \begin{align*}
    \KnuthRFoldSum{1}{n}{m} &= \sum_{j=0}^{m}
    \sum_{k=j}^{m}
    \left[
      (-1)^j \binom{t}{j+1} + \binom{j-t+n}{j+1}
    \right] \binom{t-j}{k-j} \stirlingii{m}{k} k!
  \end{align*}
\end{proposition}
By setting $t=0$ yields
\begin{proposition} [Ordinary Stirling sums of powers in zero]
  \label{prop:ordinary-stirling-sums-of-powers-in-zero}
  For non-negative integers $n,m$,
  \begin{align*}
    \KnuthRFoldSum{1}{n}{m}
    = \sum_{j=0}^{m}
    \sum_{k=j}^{m} \binom{j+n}{j+1} \binom{-j}{k-j} \stirlingii{m}{k} k!
  \end{align*}
\end{proposition}
By upper negation property of binomial coefficients
$\binom{-k}{j} = (-1)^j \binom{j+k-1}{j}$,
another binomial form for sums of powers follows.
\begin{proposition} [Ordinary Stirling sums of powers in zero altered]
  \label{prop:ordinary-stirling-sums-of-powers-in-zero-altered}
  For non-negative integers $n,m$
  \begin{align*}
    \KnuthRFoldSum{1}{n}{m}
    = \sum_{j=0}^{m} \sum_{k=j}^{m} (-1)^{k-j}
    \binom{k-1}{j-1} \binom{j+n}{j+1}  \stirlingii{m}{k} k!
  \end{align*}
\end{proposition}
Similarly to Proposition~\eqref{prop:negated-backward-sums-of-powers},
we can derive formula for double sums of powers.
Thus,
\begin{align*}
  \KnuthRFoldSum{2}{n}{m} &= \sum_{j=0}^{m} \nabla^{j} t^{m}
  \left[
    (-1)^j \binom{t}{j+1} \sum_{k=1}^{n} 1 + \sum_{k=1}^{n} \binom{j-t+k}{j+1}
  \right]
\end{align*}
By applying generalized hockey
stick identity~\eqref{lem:generalized-hockey-stick-identity},
we obtain,
\begin{align*}
  \sum_{k=1}^{n} \binom{j-t+k}{j+1}
  = \sum_{k=j-t+1}^{j-t+n} \binom{k}{j+1}
  = \binom{j-t+n+1}{j+2} - \binom{j-t+1}{j+2}.
\end{align*}
Therefore, we get closed form.
\begin{align*}
  \KnuthRFoldSum{2}{n}{m}
  &= \sum_{j=0}^{m} \nabla^{j} t^{m}
  \left[
    (-1)^j \binom{t}{j+1} n +
    \left( \binom{j-t+n+1}{j+2} - \binom{j-t+1}{j+2} \right)
  \right].
\end{align*}
By applying the identity for negative binomial
coefficients $\binom{-k}{j} = (-1)^j \binom{j+k-1}{j}$, we get,
\begin{align*}
  \binom{-(t-j-1)}{j+2} = (-1)^{j+2} \binom{t}{j+2}.
\end{align*}
Hence,
\begin{proposition} [Double sums of powers via backward differences]
  \label{prop:double-sums-of-powers-via-backward-differences}
  For non-negative integers $n,m$ and an arbitrary integer $t$,
  \begin{align*}
    \KnuthRFoldSum{2}{n}{m}
    &= \sum_{j=0}^{m} \nabla^{j} t^{m}
    \left[
      (-1)^j \binom{t}{j+1} n
      + (-1)^{j+1} \binom{t}{j+2} n^0 +  \binom{j-t+n+1}{j+2}
    \right]
  \end{align*}
\end{proposition}
By applying the hockey stick
identities~\eqref{lem:generalized-hockey-stick-identity},
and~\eqref{lem:shifted-hockey-stick-identity}
repeatedly, yields formula for $r-$fold sums of powers.
\begin{theorem} [Multifold sums of powers via backward difference]
  \label{theorem:multifold-sums-of-powers-via-backward-difference}
  For non-negative integers $r,n,m$ and an arbitrary integer $t$,
  \begin{align*}
    \KnuthRFoldSum{r}{n}{m}
    = \sum_{j=0}^{m} \nabla^{j} t^{m}
    \left[
      \binom{j-t+n+r-1}{j+r}
      + \sum_{s=0}^{r-1} (-1)^{j+s} \binom{t}{j+s+1} \KnuthRFoldSum{r-1-s}{n}{0}
    \right].
  \end{align*}
\end{theorem}
We may observe that,
\begin{proposition}[Multifold sum of zero powers]
  \label{prop:multifold-sum-of-zero-powers}
  For integers $r \geq 0$, and $n\geq 1$,
  \begin{align*}
    \KnuthRFoldSum{r}{n}{0} = \binom{r+n-1}{r}.
  \end{align*}
  \begin{proof}
    \begin{enumerate}
      \item Let $r=0$, then $\KnuthRFoldSum{0}{n}{0}=n^0=\binom{n-1}{0} = 1$,
        by definition.
      \item Let $r=1$, then
        $\KnuthRFoldSum{1}{n}{0}
        = \sum_{k=1}^{n} \binom{k-1}{0}
        = \sum_{k=1}^{n} 1 = \binom{n}{1}$.
      \item Let $r=2$, then
        $\KnuthRFoldSum{2}{n}{0}= \sum_{k=1}^{n} \binom{k}{1}
        = \sum_{k=1}^{n} k = \binom{n+1}{2}$.
      \item Let $r=3$, then
        $\KnuthRFoldSum{3}{n}{0}
        = \sum_{k=1}^{n} \binom{k+1}{2}
        = \binom{n+2}{3}$.
      \item By induction over $r$ and hockey stick identity
        $\sum_{k=r}^{n} \binom{k}{r} = \binom{n+1}{r+1}$,
        the claim follows
        $\KnuthRFoldSum{r}{n}{0} = \binom{r+n-1}{r}$.
    \end{enumerate}
    This completes the proof.
  \end{proof}
\end{proposition}
Hence, binomial form of formula for multifold sums of powers
~\eqref{theorem:multifold-sums-of-powers-via-backward-difference}
follows.
\begin{proposition} [Multifold sums of powers binomial form]
  \label{prop:multifold-sums-of-powers-via-backward-difference-binomial-form}
  For non-negative integers $r,n,m$ and an arbitrary integer $t$,
  \begin{align*}
    \KnuthRFoldSum{r}{n}{m} = \sum_{j=0}^{m} \nabla^{j} t^{m}
    \left[
      \binom{j-t+n+r-1}{j+r}
      + \sum_{s=0}^{r-1} (-1)^{j+s} \binom{t}{j+s+1} \binom{r-s+n-2}{r-s-1}
    \right].
  \end{align*}
\end{proposition}
By setting $r \rightarrow r+1$ yields shifted version of the identity above.
\begin{corollary} [Multifold sums of powers binomial form shifted]
  \label{cor:multifold-sums-of-powers-via-backward-difference-binomial-form-altered}
  For non-negative integers $r,n,m$ and an arbitrary integer $t$,
  \begin{align*}
    \KnuthRFoldSum{r+1}{n}{m} = \sum_{j=0}^{m} \nabla^{j} t^{m}
    \left[
      \binom{j-t+n+r}{j+r+1}
      + \sum_{s=0}^{r} (-1)^{j+s} \binom{t}{j+s+1} \binom{r-s+n-1}{r-s}
    \right].
  \end{align*}
\end{corollary}
By Lemma~\eqref{lem:backward-differences-in-stirling-numbers},
we get formula for multifold sums of powers
in terms of Stirling numbers of the second kind $\stirlingii{n}{k}$.
\begin{proposition}[Multifold sums of powers in Stirling numbers]
  \label{prop:multifold-sums-of-powers-in-stirling-numbers}
  For non-negative integers $r,n,m$ and an arbitrary integer $t$,
  \begin{align*}
    \KnuthRFoldSum{r+1}{n}{m}
    = \sum_{j=0}^{m} \sum_{k=j}^{m}
    \left[
      \tbinom{j-t+n+r}{j+r+1}
      + \sum_{s=0}^{r} (-1)^{j+s} \tbinom{t}{j+s+1} \tbinom{r-s+n-1}{r-s}
    \right] \tbinom{t-j}{k-j} \tstirlingii{m}{k} k!
  \end{align*}
\end{proposition}
By lemma~\eqref{lem:backward-differences-in-eulerian-numbers},
we get formula for multifold sums of powers
in terms of Eulerian numbers $\eulerian{n}{k}$.
\begin{proposition}[Multifold sums of powers in Eulerian numbers]
  \label{prop:multifold-sums-of-powers-in-eulerian-numbers}
  For non-negative integers $r,n,m$ and an arbitrary integer $t$
  \begin{align*}
    \KnuthRFoldSum{r+1}{n}{m}
    = \sum_{j=0}^{m} \sum_{k=0}^{m}
    \left[
      \tbinom{j-t+n+r}{j+r+1}
      + \sum_{s=0}^{r} (-1)^{j+s} \tbinom{t}{j+s+1} \tbinom{r-s+n-1}{r-s}
    \right] \tbinom{t+k-j}{m-j} \teulerian{m}{k}.
  \end{align*}
\end{proposition}



    \section*{Conclusions}
    In this manuscript, we derived formulas for multifold sums of
powers~\eqref{theorem:multifold-sums-of-powers-via-backward-difference},
by utilizing Newton's interpolation series in backward differences.
In addition, we highlight that backward differences of powers are closely
related to
Stirling numbers of the second kind $\stirlingii{n}{k}$,
and Eulerian numbers $\eulerian{m}{k}$.
Thus, we expressed  formulas for multifold sums of powers in terms of Stirling numbers
of the second kind~\eqref{prop:multifold-sums-of-powers-in-stirling-numbers}
and Eulerian numbers~\eqref{prop:multifold-sums-of-powers-in-eulerian-numbers}.
Future research directions are discussed and proposed
in~\cite{kolosov_2025_18102534}, which includes
development of generalized framework for sums of powers using interpolation
formulas combined with
hockey stick identities for binomial coefficients.
All the results are validated using \texttt{Mathematica}
programs; see Section~\eqref{sec:mathematica-programs}.


    \bibliographystyle{unsrt}
    \bibliography{SumsOfPowersViaBackwardFiniteDifferencesAndNewtonFormula}

    \clearpage

    \section{Mathematica programs}
    \label{sec:mathematica-programs}
    Use the \textit{Mathematica} package~\cite{MathematicaPrograms} to validate the results
\begin{center}
  \renewcommand{\arraystretch}{1.3}
  \begin{tabular}{ll}
    \toprule
    \textbf{Mathematica Function}                                     & \textbf{Validates / Prints}                                                                       \\
    \midrule
    \texttt{MultifoldSumOfPowersRecurrence[r, n, m]}
    & Computes $\KnuthRFoldSum{r}{n}{m}$                                                                \\
    \texttt{ValidateOrdinarySumsOfPowersViaBackwardDifferences[20]}
    & Validates~\eqref{prop:ordinary-sums-of-powers-via-backward-differences}               \\
    \texttt{ValidateBackwardDifferencesInEulerianNumbers[20]}
    & Validates~\eqref{lem:backward-differences-in-eulerian-numbers}                              \\
    \texttt{ValidateOrdinarySumsOfPowersInEulerianNumbers[10]}
    & Validates~\eqref{prop:ordinary-sums-of-powers-in-eulerian-numbers}                    \\
    \texttt{ValidateBackwardDifferencesInStirlingNumbers[20]}
    & Validates~\eqref{lem:backward-differences-in-stirling-numbers}                              \\
    \texttt{ValidateOrdinarySumsOfPowersInStirlingNumbers[20]}
    & Validates~\eqref{prop:ordinary-sums-of-powers-in-stirling-numbers}                    \\
    \texttt{ValidateOrdinaryStirlingSumsOfPowersInZero[20]}
    & Validates~\eqref{prop:ordinary-stirling-sums-of-powers-in-zero}                      \\
    \texttt{ValidateOrdinaryStirlingSumsOfPowersInZeroAltered[20]}
    & Validates~\eqref{prop:ordinary-stirling-sums-of-powers-in-zero-altered}               \\
    \texttt{ValidateDoubleSumsOfPowersViaBackwardDifferences[10]}
    & Validates~\eqref{prop:double-sums-of-powers-via-backward-differences}                 \\
    \texttt{ValidateMultifoldSumsOfPowersViaBackwardDifference[5]}
    & Validates~\eqref{theorem:multifold-sums-of-powers-via-backward-difference}                \\
    \texttt{ValidateMultifoldSumsOfPowersBackwardDiffBinomialForm[5]}
    & Validates~\eqref{prop:multifold-sums-of-powers-via-backward-difference-binomial-form} \\
    \texttt{ValidateMultifoldSumsOfPowersBinomialFormShifted[5]}
    & Validates~\eqref{cor:multifold-sums-of-powers-via-backward-difference-binomial-form-altered} \\
    \texttt{ValidateMultifoldSumsOfPowersInStirlingNumbers[5]}
    & Validates~\eqref{prop:multifold-sums-of-powers-in-stirling-numbers} \\
    \texttt{ValidateMultifoldSumsOfPowersInEulerianNumbers[5]}
    & Validates~\eqref{prop:multifold-sums-of-powers-in-eulerian-numbers} \\
    \bottomrule
  \end{tabular}
\end{center}


    \clearpage

    {\large\textbf{Metadata}}
\begin{itemize}
    \item \textbf{Version:} \input{metadata/version}
    \item \textbf{MSC2010:} 05A19, 05A10, 11B68, 11B73, 11B83
    \item \textbf{Keywords:} \input{metadata/footer-keywords.tex}
    \item \textbf{License:} This work is licensed under a \href{https://creativecommons.org/licenses/by/4.0/}{\texttt{CC BY 4.0 License}}.
    \item \textbf{DOI:} \href{https://doi.org/10.5281/zenodo.18118011}{\texttt{https://doi.org/10.5281/zenodo.18118011}}
    \item \textbf{Web Version:} \href{https://kolosovpetro.github.io/sums-of-powers-backward-differences/}{\texttt{kolosovpetro.github.io/sums-of-powers-backward-differences/}}
    \item \textbf{Sources:} \href{https://github.com/kolosovpetro/SumsOfPowersViaBackwardFiniteDifferencesAndNewtonFormula}{\texttt{github.com/kolosovpetro/SumsOfPowersViaBackwardFiniteDifferencesAndNewtonFormula}}
    \item \textbf{ORCID:} \href{https://orcid.org/0000-0002-6544-8880}{\texttt{0000-0002-6544-8880}}
    \item \textbf{Email:} \href{mailto:kolosovp94@gmail.com}{\texttt{kolosovp94@gmail.com}}
\end{itemize}


\end{document}
